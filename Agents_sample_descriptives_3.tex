% Options for packages loaded elsewhere
\PassOptionsToPackage{unicode}{hyperref}
\PassOptionsToPackage{hyphens}{url}
%
\documentclass[
]{article}
\usepackage{lmodern}
\usepackage{amssymb,amsmath}
\usepackage{ifxetex,ifluatex}
\ifnum 0\ifxetex 1\fi\ifluatex 1\fi=0 % if pdftex
  \usepackage[T1]{fontenc}
  \usepackage[utf8]{inputenc}
  \usepackage{textcomp} % provide euro and other symbols
\else % if luatex or xetex
  \usepackage{unicode-math}
  \defaultfontfeatures{Scale=MatchLowercase}
  \defaultfontfeatures[\rmfamily]{Ligatures=TeX,Scale=1}
\fi
% Use upquote if available, for straight quotes in verbatim environments
\IfFileExists{upquote.sty}{\usepackage{upquote}}{}
\IfFileExists{microtype.sty}{% use microtype if available
  \usepackage[]{microtype}
  \UseMicrotypeSet[protrusion]{basicmath} % disable protrusion for tt fonts
}{}
\makeatletter
\@ifundefined{KOMAClassName}{% if non-KOMA class
  \IfFileExists{parskip.sty}{%
    \usepackage{parskip}
  }{% else
    \setlength{\parindent}{0pt}
    \setlength{\parskip}{6pt plus 2pt minus 1pt}}
}{% if KOMA class
  \KOMAoptions{parskip=half}}
\makeatother
\usepackage{xcolor}
\IfFileExists{xurl.sty}{\usepackage{xurl}}{} % add URL line breaks if available
\IfFileExists{bookmark.sty}{\usepackage{bookmark}}{\usepackage{hyperref}}
\hypersetup{
  pdftitle={\#Agents: Sample Statistics 2020},
  hidelinks,
  pdfcreator={LaTeX via pandoc}}
\urlstyle{same} % disable monospaced font for URLs
\usepackage[margin=1in]{geometry}
\usepackage{graphicx,grffile}
\makeatletter
\def\maxwidth{\ifdim\Gin@nat@width>\linewidth\linewidth\else\Gin@nat@width\fi}
\def\maxheight{\ifdim\Gin@nat@height>\textheight\textheight\else\Gin@nat@height\fi}
\makeatother
% Scale images if necessary, so that they will not overflow the page
% margins by default, and it is still possible to overwrite the defaults
% using explicit options in \includegraphics[width, height, ...]{}
\setkeys{Gin}{width=\maxwidth,height=\maxheight,keepaspectratio}
% Set default figure placement to htbp
\makeatletter
\def\fps@figure{htbp}
\makeatother
\setlength{\emergencystretch}{3em} % prevent overfull lines
\providecommand{\tightlist}{%
  \setlength{\itemsep}{0pt}\setlength{\parskip}{0pt}}
\setcounter{secnumdepth}{-\maxdimen} % remove section numbering
\usepackage{flafter}
\usepackage{float}
\usepackage{booktabs}
\usepackage{longtable}
\usepackage{array}
\usepackage{multirow}
\usepackage{wrapfig}
\usepackage{colortbl}
\usepackage{pdflscape}
\usepackage{tabu}
\usepackage{threeparttable}
\usepackage{threeparttablex}
\usepackage[normalem]{ulem}
\usepackage{makecell}
\usepackage{xcolor}

\title{\#Agents: Sample Statistics 2020}
\author{}
\date{\vspace{-2.5em}}

\begin{document}
\maketitle

This is a documentation of the \#Agents -data. The data was collected in
between 9.12.2019 -- 19.1.2020 as a telephone survey from a gender and
age - balanced sample of 15-19 year-old finnish adolescents. The total
sample size is 800. An average interview took \textasciitilde30 minutes
and the sample was drawn from the civil registry of Finland. All
together 24269 phone numbers were contacted during the sampling period.

\includegraphics{Agents_sample_descriptives_3_files/figure-latex/unnamed-chunk-6-1}
Regarding age or gender the over/underrepresentations were less than 1\%
(see also Table 2).

\newpage

\hypertarget{missing-values}{%
\section{1. Missing values}\label{missing-values}}

The data were collected using a self-report questionnaire. There were
0.168\% of the data missing altogether (not including open ended
questions), none of the variables showed more than 5\% missing.

\includegraphics{Agents_sample_descriptives_3_files/figure-latex/unnamed-chunk-7-1.pdf}

The most missing (2.62\% to 3.75\%) were in items assessing issues
related to income (q280,q281,q282) and item q135 assessing
trustworthiness of a known fake-news outlet. The income items also
showed the most common intersections (same items missing from the same
participants).

\includegraphics{Agents_sample_descriptives_3_files/figure-latex/unnamed-chunk-9-1.pdf}

Based on the non-parametric test of heteroscedasticity, the assumption
of missing completely at random (MCAR) was rejected for the numeric data
(p = 0.001), but the overall amount of missing data in the dataset is
very small.

\hypertarget{geographical-distribution-of-participants}{%
\section{2. Geographical distribution of
participants}\label{geographical-distribution-of-participants}}

\includegraphics{Agents_sample_descriptives_3_files/figure-latex/unnamed-chunk-12-1.pdf}

There were participants from 554 unique zip codes, of which 525 were
valid. In total 770 participants gave a valid zip code (i.e.~30
participants had reported a incorrect zip code, see also Table 1). The
zip codes were from 184 different municipalities from all states in
Finland\footnote{geofi package was used: Markus Kainu, Joona Lehtomäki,
  Juuso Parkkinen, Jani Miettinen, Leo Lahti Retrieval and analysis of
  open geospatial data from Finland with the geofi R package. R package
  version 0.9.2900006. URL: \url{http://ropengov.github.io/geofi}}.

\newpage

\hypertarget{current-status}{%
\section{3. Current status}\label{current-status}}

\includegraphics{Agents_sample_descriptives_3_files/figure-latex/unnamed-chunk-15-1}
Most of the participants current status (see also Table 3) were either
comprehensive school or secondary education student, who still lived
their parents (Table 4).

\newpage

\hypertarget{socio-economic-status}{%
\section{4. Socio-economic status}\label{socio-economic-status}}

\includegraphics{Agents_sample_descriptives_3_files/figure-latex/unnamed-chunk-18-1}

Examination of the participants socio-economic status indicated that the
sample comprised of a representative distribution of young people from
low to high socio-economic status. The majority reported that their
mother had acquired at least a secondary degree and were in an adequate
to good financial situation (see also Tables 5 and 6).

\newpage

\hypertarget{tables}{%
\section{Tables}\label{tables}}

\begin{table}

\caption{\label{tab:region-table}Regional distribution of Participants}
\centering
\begin{tabular}[t]{>{}l|r|r}
\hline
Region & Participants & \%\\
\hline
\textbf{Ahvenanmaa} & 1 & 0.00\\
\hline
\textbf{Etelä-Karjala} & 16 & 0.02\\
\hline
\textbf{Etelä-Pohjanmaa} & 41 & 0.05\\
\hline
\textbf{Etelä-Savo} & 27 & 0.04\\
\hline
\textbf{Kainuu} & 15 & 0.02\\
\hline
\textbf{Kanta-Häme} & 33 & 0.04\\
\hline
\textbf{Keski-Pohjanmaa} & 13 & 0.02\\
\hline
\textbf{Keski-Suomi} & 55 & 0.07\\
\hline
\textbf{Kymenlaakso} & 18 & 0.02\\
\hline
\textbf{Lappi} & 18 & 0.02\\
\hline
\textbf{Pirkanmaa} & 78 & 0.10\\
\hline
\textbf{Pohjanmaa} & 9 & 0.01\\
\hline
\textbf{Pohjois-Karjala} & 23 & 0.03\\
\hline
\textbf{Pohjois-Pohjanmaa} & 86 & 0.11\\
\hline
\textbf{Pohjois-Savo} & 38 & 0.05\\
\hline
\textbf{Päijät-Häme} & 27 & 0.04\\
\hline
\textbf{Satakunta} & 30 & 0.04\\
\hline
\textbf{Uusimaa} & 168 & 0.22\\
\hline
\textbf{Varsinais-Suomi} & 74 & 0.10\\
\hline
\end{tabular}
\end{table}

\begin{table}

\caption{\label{tab:sex-age-table}Age and Gender distribution}
\centering
\begin{tabular}[t]{l|c|c|c|c|c|c}
\hline
Sex & Value & 15 Years & 16 Years & 17 Years & 18 Years & 19 Years\\
\hline
\textbf{Male} & \textbf{n} & \textbf{78} & \textbf{83} & \textbf{80} & \textbf{79} & \textbf{79}\\
\hline
 & \% & 10 & 10 & 10 & 10 & \vphantom{1} 10\\
\hline
\textbf{Female} & \textbf{n} & \textbf{77} & \textbf{84} & \textbf{79} & \textbf{79} & \textbf{82}\\
\hline
 & \% & 10 & 10 & 10 & 10 & 10\\
\hline
\end{tabular}
\end{table}

\begin{table}

\caption{\label{tab:surrent-status-table}Educational level (or occupational status)}
\centering
\begin{tabular}[t]{l|c|c|c|c|c}
\hline
Level & Value & Studying & Working & Army/civil service & Other\\
\hline
\textbf{Not studying} & \textbf{n} & \textbf{0} & \textbf{64} & \textbf{14} & \textbf{23}\\
\hline
 & \% & 0 & 8 & 2 & 3\\
\hline
\textbf{Comprehensive school} & \textbf{n} & \textbf{177} & \textbf{0} & \textbf{0} & \textbf{0}\\
\hline
 & \% & 22 & 0 & 0 & 0\\
\hline
\textbf{High school} & \textbf{n} & \textbf{313} & \textbf{0} & \textbf{0} & \textbf{0}\\
\hline
 & \% & 39 & 0 & 0 & 0\\
\hline
\textbf{Vocational school} & \textbf{n} & \textbf{166} & \textbf{0} & \textbf{0} & \textbf{0}\\
\hline
 & \% & 21 & 0 & 0 & 0\\
\hline
\textbf{High school + vocational} & \textbf{n} & \textbf{18} & \textbf{0} & \textbf{0} & \textbf{0}\\
\hline
 & \% & 2 & 0 & 0 & \vphantom{1} 0\\
\hline
\textbf{Applied college} & \textbf{n} & \textbf{14} & \textbf{0} & \textbf{0} & \textbf{0}\\
\hline
 & \% & 2 & 0 & 0 & 0\\
\hline
\textbf{University} & \textbf{n} & \textbf{11} & \textbf{0} & \textbf{0} & \textbf{0}\\
\hline
 & \% & 1 & 0 & 0 & 0\\
\hline
\end{tabular}
\end{table}

\begin{table}

\caption{\label{tab:living-table}Living arrangements}
\centering
\begin{tabular}[t]{>{}l|c|c|c|c}
\hline
  & w/ Partner & w/ Parents & Independently & Other\\
\hline
\textbf{n} & 16 & 668 & 110 & 6\\
\hline
\textbf{\%} & 2 & 84 & 14 & 1\\
\hline
\end{tabular}
\end{table}

\begin{table}

\caption{\label{tab:mom-edu-table}Mother's educational level}
\centering
\resizebox{\linewidth}{!}{
\begin{tabular}[t]{>{}l|c|c|c|c|c|c|c}
\hline
  & Comprehensive School & High School & Vocational School & Vocational College & Applied University & University & Unknown\\
\hline
\textbf{n} & 16 & 85 & 155 & 37 & 146 & 168 & 192\\
\hline
\textbf{\%} & 2 & 11 & 19 & 5 & 18 & 21 & 24\\
\hline
\end{tabular}}
\end{table}

\begin{table}

\caption{\label{tab:fin-sit-table}Financial situation}
\centering
\begin{tabular}[t]{l|c|c|c|c|c|c}
\hline
Variable & Value & Very Poor & Poor & Adequate & Good & Very Good\\
\hline
\textbf{Family} & \textbf{n} & \textbf{1} & \textbf{30} & \textbf{203} & \textbf{414} & \textbf{152}\\
\hline
 & \% & 0 & 4 & 25 & 52 & 19\\
\hline
\textbf{Personal} & \textbf{n} & \textbf{7} & \textbf{47} & \textbf{208} & \textbf{416} & \textbf{119}\\
\hline
 & \% & 1 & 6 & 26 & 52 & 15\\
\hline
\end{tabular}
\end{table}

\end{document}
